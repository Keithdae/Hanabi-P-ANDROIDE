\documentclass{article}
\usepackage[utf8]{inputenc}

\title{Approches sur la modélisation du jeu Hanabi}
\date{}
\usepackage{natbib}
\usepackage{graphicx}

\begin{document}

\maketitle

\section{Modélisation exhaustive d'une situation}


    Ici le but est d'identifier chaque situation de jeu possible par un monde de Kripke.
La première étape est de modéliser les mondes liés à la main (combinaison de 4 ou 5 cartes sans
ordre) du joueur, c'est à dire toutes les mains possibles en prenant en compte les cartes vues, déjà jouées ou déjà défaussés. L'agent correspondant au joueur hésite donc entre tous ces mondes possibles.
Notre structure de Kripke est donc:
\smallbreak
$   \mu=\{U,\{R_{1}\},I \}  $
$   U=\{M_{1},...M_{m} \}   $

$R$: A ce stade, $R_{1}$ connecte tous les mondes, on suppose que l'on ne génére pas les mondes qu'il sait déjà impossibles.

$I$: On note    $C_{j,c,v}$ la variable affirmant que le joueur j possède une carte de couleur c et de valeur v. Pour n joueurs, un monde $M_{i}$ contiendra donc toujours un nombre de variables $nv=n*4 | n>3$ ou $nv=n*5 | n<4$
\smallbreak
A ce niveau de modélisation, on peut déjà envisager des prises de décisions probabilistes simplement en comptant le nombre de mondes dans lesquels faire telle action aurait un impact positif.

Ensuite, il est également possible pour le joueur de modéliser les mondes entre lesquels il pense qu'un autre joueur hésite. Appelons le joueur qui réfléchit $j1$ et celui sur lequel il porte sa réflexion $j2$. Cela pose problème car ce $j2$ dispose d'informations que n'a pas $j1$ (ses propres cartes), mais on peut tout de même savoir que des mondes sont impossibles grâce à l'information commune aux 2 joueurs (les cartes des autres joueurs, les cartes jouées, les cartes défaussées, les indices donnés).
Pour cela, on peut se baser sur les formules de la logique propositionnelle, auxquelles on ajoute la modalité de la connaissance: $[]_j$ Les atomes sont représentés sous la forme :$\Phi_{j,c,c1,v1}$ tels que "la carte $c$ du joueur $j$ a la couleur $c1$ et la valeur $v1$". L'utilisation de l'axiome de connaissance est donc   $[]_j F$ est vrai dans un monde $M$ si tous les $M'$ successeurs de $M$ pour la relation $R_j$ satisfont $F$. Ce raisonnement peut être appliqué pour chaque autre joueur.
On a donc maintenant:

$   \mu'=\{U',\{R_{1},...,R_{n}\},I' \} $

$   U'=\{M_{1},...M_{m'} \} $

\newpage
Pour les relations $R_{i}$, il faut se dire que le joueur $i$ n'hésite qu'entre les mondes où toutes les cartes sont identiques sauf les siennes.
\smallbreak
$   \forall i \in [2,...,n] ,  \forall (M_{j},M_{k}) \in U'*U',$
\smallbreak
$R_{i}(M_{j},M_{k}) 
\equiv
\forall l \in [1,...,n]|l \ne i,  \forall C_{l,c1,v1} \in  I(M_{j}), \forall C_{l,c2,v2} \in  I(M_{k}), c1=v1, c2=v2    $

Ensuite, il faut tenir compte des indices donnés par les joueurs qui sont des "annonces publiques" qui viennent enrichir la connaissance commune à tous les joueurs. Grâce à ces annonces et aux formules proposées plus haut, il est possible de supprimer des membres des relations (graphiquement, cela correspond à supprimer des arrêtes du graphe), et donc certains monde deviennent impossibles.

\section{Modélisation carte par carte}

Cette fois, l'objectif est de créer des structures de Kripke pour chaque carte plutôt que pour chaque situation. Ainsi, sans information, il y aurait 25 mondes pour chaque carte. On peut ensuite tenter d'appliquer les mêmes raisonnements que dans la modélisation exhaustive.

Univers de la i-ème carte du joueur $j$:

$   \mu_{i,j}=\{U,\{R_{1},...,R_{n}\},I \}  $

$   U=\{M_{1},...M_{m} \}   $

$I$: chaque monde contient 2 variables, une pour la couleur et une pour la valeur. 

$\forall M \in U, I(M)=(c,v)|c\in[rouge,bleu,vert,jaune,blanc],v\in[1,...,5]$
\smallbreak
Cette approche peut être utilisée comme la précédente pour choisir les coups à jouer de manière probabiliste. Cependant, si l'on veut essayer de raisonner sur la connaissance des joueurs, il est nécessaire de mettre en relation tous ces univers, et on en revient donc à la modélisation exhaustive. Cette approche n'est donc pas réellement intéressante,puisque il est possible d'utiliser des raisonnement probabilistes sans structures de Kripke. 
\end{document}
